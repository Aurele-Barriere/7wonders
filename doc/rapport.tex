\documentclass[12pt]{article}
\usepackage[utf8]{inputenc}
\usepackage[T1]{fontenc}
\usepackage[french]{babel}
\usepackage{amsmath,amsfonts,amssymb}
\usepackage{fullpage}
\usepackage{graphicx}

\def\question#1{\subsection*{#1}}
\def\sec#1{\section*{#1}}

\title{Les 7 merveilleuses merveilles du merveilleux monde des 7 merveilleuses couleurs}
\author{Aurèle Barrière \& Nathan Tomasset}
\date{8 mars 2016}

\begin{document}
\maketitle

\sec{Intoduction}
Nous présentons dans ce projet une implémentation du jeu des 7 couleurs dont les règles seront décrites plus bas. L'enjeu est d'implémenter en C le jeu, puis de créer des stratégies pour jouer.
%à compléter

\question{Règles du jeu}
Le jeu se présente comme un tableau carré de 30 case de côté (ce nombre est paramétrable dans notre implémentatoin). Initialment, il est rempli aléaoirement avec 7 ``couleurs'' (ou lettres). Dans deux angles opposés, il y a une huitième et une neuvième couleurs : celles des joueurs. Tour à tour, les joueurs choisissent une couleur parmi les 7. Toutes les cases de cette couleur et adjacente à la couleur du joueur (ou adjacente à une case qui vient de changer de couleur à ce tour par cette méthode) prennent désormais la couleur du joueur. Le but est de posséder plus de la moitié du plateau.

\sec{Organisation du projet}

%schéma objet, structures utilisées, ...


\sec{Voir le monde en 7 couleurs}
\question{2.1} 
Pour initialiser le tableau de manière aléatoire, on parcourt la matrice (initialisée avec des 0), et on remplace chaque coefficient par une des 7 couleurs.

Pour obtenir ce nombre, on importe le module \texttt{stdlib}, on initialise le générateur pseudo-aléatoire avec le temps (\texttt{srand(time(NULL));}) et on prend le résultat de la fonction \texttt{rand()} modulo le nombre de couleurs.

Cependant, il faut bien initialiser l'aléatoire au début de la fonction principale, et pas à chaque partie. En effet, dans la suite on lancera plein de parties en même temps, qui seront donc identiques si lancées à moins d'une seconde d'intervalle.

Enfin,\texttt{rand()} modulo un nombre de couleurs ne produit pas un résultat avec une probabilité parfaitement uniforme. %a detailler
Cependant, vu le petit nombre de couleurs, ce n'est pas un problème significatif.

\question{2.2}
On créée une fonction \texttt{void update\_board(char player, char color, char * b)}. Elle prend en argument un joueur \texttt{player}, une couleur jouée \texttt{color} et un plateau \texttt{b}.

Elle consiste à parcourir la matrice, et si on trouve une case qui est de la couleur jouée et à côté d'une case de la couleur du joueur, on la remplace et on indique dans une variable qu'il y a eu un changement.
S'il y a eu changement, il faut recommencer le parcours de la matrice.

On pose $n$ la taille du tableau. Dans le pire cas, il faut réappliquer $n$ fois le parcours de matrice (à chaque parcours, il y a un changement). On a donc une complexité en $\mathcal{O}(n^3)$ dans le pire cas.
%il faut détailler

\question{2.3}
%TO DO

\sec{À la conquête du monde}
\question{3.1}
On crée donc une fonction \texttt{player\_choice}, qui demande à un joueur un caractère correspondant à une couleur. Si le caractère ne correspond pas à une de couleurs disponibles, on redemande jusqu'à obtenir un résultat convenable.

%limites? qu'est ce qu'il va pas elle pas belle ma fonction?

\question{3.2}
On implémente la fonction \texttt{int score (char * b, int color)} qui prend en argument un plateau de jeu et la couleur d'un joueur. elle se content de compter le nombre de cases de cette couleur pour calculer le score d'un joueur.

On calcule la limite de score à atteindre : si $n$ est la taille du plateau, on prend $\frac{n^2}{2}$ (arrondi à la valeur supérieure).

Grâce à cette fonction, on peut donc vérifier à chaque tour si :
\begin{itemize}
\item un des joueurs a atteint la limite de score
\item les deux joueurs ont le même score qui est égal à la limite (égalité)
\end{itemize}

Ainsi, on peut mettre des conditions d'arrêt au jeu.
On peut également calculer le pourcentage en calculant $\frac{score \times 100}{n^2}$.



\end{document}
